\documentclass{article}
\usepackage[utf8]{inputenc}
\usepackage{amsmath}
\usepackage{amssymb}
\usepackage{fancyvrb} % for the Verbatim environment

% Define the pseudocode environment
\fvset{fontsize=\small} % smaller font for the pseudocode
\newenvironment{pseudocode}
{\begin{Verbatim}[commandchars=\\\{\}]}
{\end{Verbatim}}

\title{Calculating Series Y}
\author{Allie A. Phenish}
\date{\today}

\begin{document}

\maketitle

\section{Problem Description}

The problem is to calculate the value of the series \textbf{Y}:
$$
Y = \frac{X_1}{2!} + \frac{X_2}{4!} + \frac{X_3}{8!} + \dots + \frac{X_n}{(2^n)!}
$$

This can be expressed using summation notation as:
$$
Y = \sum_{i=1}^{n} \frac{X_i}{(2^i)!}
$$

\section{Pseudocode}

Here is a concise pseudocode solution that uses a separate function for the factorial calculation.

\begin{pseudocode}
procedure Calculate_Y(X_list, n)
    sum = 0
    for j from 1 through n
        term = X_list[j] / factorial(2^j)
        sum = sum + term
    return sum

function factorial(n)
    if n is 0 or 1
        return 1
    else
        return n * factorial(n - 1)
\end{pseudocode}

We assumes 2\verb|^|j is an atomic operation, which is a reasonable assumption for high-level pseudocode.

\end{document}