\documentclass{article}
\usepackage[utf8]{inputenc}
\usepackage{amsmath}
\usepackage{amssymb}
\usepackage{fancyvrb} % for the Verbatim environment

% Define the pseudocode environment
\fvset{fontsize=\small} % smaller font for the pseudocode
\newenvironment{pseudocode}
{\begin{Verbatim}[commandchars=\\\{\}]}
{\end{Verbatim}}

\title{Calculating Product Z}
\author{Allie A. Phenish}
\date{\today}

\begin{document}

\maketitle

\section{Problem Description}

The problem is to calculate the value of the series \textbf{Z}, which is a product of four terms:
$$Z = (3 \cdot X_1) \cdot (6 \cdot X_2) \cdot (12 \cdot X_3) \cdot (24 \cdot X_4)$$
This can be expressed using product notation as:
$$Z = \prod_{i=1}^{4} (3 \cdot 2^{i-1} \cdot X_i)$$

\section{Pseudocode}

A concise solution is presented in the pseudocode below. The solution uses a loop to iterate through each term of the series, calculating the coefficient for each $X_i$ and multiplying it by the corresponding value.

\begin{pseudocode}
procedure Calculate_Z(X_list)
    product = 1
    for j from 1 through 4
        coefficient = 3 * pow(2, j-1)
        term = coefficient * X_list[j]
        product = product * term
    return product
\end{pseudocode}

We assume \texttt{pow(x, y)} is an atomic operation, which is a reasonable assumption for high-level pseudocode.

\end{document}
